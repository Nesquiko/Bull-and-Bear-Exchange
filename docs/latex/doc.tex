\documentclass[11pt,a4paper]{article}

\usepackage[USenglish]{babel}
\usepackage[IL2]{fontenc}
\usepackage[utf8]{inputenc}
\usepackage{graphicx}
\usepackage{url}
\usepackage{cite}
\usepackage{listings}
\usepackage{color}
\usepackage{float}
\usepackage{subfiles}
\usepackage{lipsum}
\usepackage[outputdir=build]{minted}

\usepackage[colorlinks = true,
            linkcolor = blue,
            urlcolor  = blue,
            citecolor = blue,
            anchorcolor = blue]{hyperref}

\newcounter{question}
\setcounter{question}{0}
\newcommand\Que[1]{
   \leavevmode\par
   \stepcounter{question}
   \noindent
   \thequestion. #1\par}

\newcommand\Ans[2][]{
    \leavevmode\par\noindent
   {\leftskip37pt
    A --- \textbf{#1}#2\par}}

\raggedbottom
\oddsidemargin=0cm
\evensidemargin=0cm
\textwidth=16.5cm
\pagestyle{plain}

\title{Bull and Bear Exchange\\
    \large DMBLOCK Assignment 2}

\author{Lukáš Častven, Jakub Jelínek\\[2pt]
	{\small Slovenská technická univerzita v Bratislave}\\
	{\small Fakulta informatiky a informačných technológií}\\
	{\small \texttt{xcastven@stuba.sk, xjelinek@stuba.sk}}
}

\date{\today}

\begin{document}
\pagestyle{plain}

\maketitle
\tableofcontents
\pagebreak

\section{Assignment}

Main goal of this assignment was to complete provided implementation of an
Uniswap\cite{uniswapDocs} inspired decentralized exchange. We were given
solidity and javascript source codes in a Hardhat project, and we had to
implement methods in these files to create a functional decentralized exchange
for swapping Ether with our custom ERC20 token.

\section{Questions}

\Que{Why removing liquidity from exchange doesn't change the rate}

\Que{Explain implemented fee mechanism for incentivizing liquidity providers}

\Que{Explain at least one gas optimisation method you used}

\subsection*{Feedback questions}

\Que{How much time did you spend on the assignment}

\Que{What would one useful information before you started to work on this assignment}

\Que{What would one thing you would change}

\section{Implementation}

We decided to rewrite the provided Hardhat project into Foundry\cite{foundryBook}.
Our decentralized exchange is called \textbf{Bull \& Bear Exchange} and the ERC20
token traded on this exchange is \textbf{Bull \& Bear Token}. Also we rewrote
provided web app into Vue\cite{vuejs}.

The project structure looks like this:

\begin{itemize}
    \item \texttt{app} - contains the Vue frontend interacting with the smart contracts
    \item \texttt{dex} - contains the Foundry project for the smart contracts of \textbf{Bull \& Bear Exchange}
    \item \texttt{docs} - contains documentation for this assignment
\end{itemize}

\subsection*{Smart contracts}

\subsubsection*{Bull \& Bear Token --- BBT}

ERC20 token to be traded on our exchange is called \textbf{Bull \& Bear token},
with symbol being \textbf{BBT}. We argue that the two functions from assignment
(\texttt{mint} and \texttt{disable\_mint}), which we have to implement, are
useless and potentially an anti-pattern. ERC20 implementation by
OpenZeppelin is enough to implement a token with constant supply.
By pre-minting supply to the deployer of the token, we achieved a token
with constant supply (no new tokens can be minted as \texttt{\_mint} in ERC20
is an internal function\cite{openzeppelinERC20}). Thus we have reduced the
complexity of this token implementation by removing \texttt{mint}, \texttt{disable\_mint}
and even the \texttt{Ownable} parent contract used in provided source code.

Thus the whole contract has few lines and minimal complexity:

\begin{minted}{solidity}
import {ERC20} from "@openzeppelin/contracts/token/ERC20/ERC20.sol";

contract BBToken is ERC20 {
    constructor(uint256 supply) ERC20("Bull and Bear Token", "BBT") {
        _mint(msg.sender, supply * 10 ** decimals());
    }

    function decimals() public pure override returns (uint8) {
        return 0;
    }
}
\end{minted}

The assignment requires that our token be indivisible, the function
decimals is overridden to reflect this requirement.

\section{Testing}

\section{Security analysis}

\section{Our improvements}

We have made several improvements to the original implementation of the decentralized exchange.
First of all, we have used Foundry instead of Hardhat, which allowed us to write the smart contracts in a more concise way.
We have also rewritten the web application in Vue.js, which is a more modern and user-friendly framework than the original jQuery-based \quote application \quote.

\section{Conclusion}

In this assignment, we have learned how to create and implement a decentralized exchange inspired by Uniswap v1.
Thanks to that, we were able to implement the smart contracts for the exchange and the ERC20 token, using Solidity and Foundry.
These contracts were then connected to a web application rewritten in Vue.js, which allows users to interact with the exchange.

\bibliography{bibliography}
\bibliographystyle{ieeetr}

\end{document}
